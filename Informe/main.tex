\documentclass[screen, pagebackref,oneside]{ibtesis}
\usepackage{subfiles}
\usepackage{preamble}

%%%%%%%%%%%%%%%%%%%%%%%%%%%%%%%%%%%%%%%%%%%%%%%%%%%%%%%%%%%%%%%%%%%%%%%%%%%%%%%%
%%%%%%%%%%%%%%%%%%%%% Informacion sobre la tesis %%%%%%%%%%%%%%%%%%%%%%%%%%%%%%%
    \title{Receptor para estación terrena con conformación digital de haz}
    \author{Francisco Gabriel Marquinez}
    \director{Ing. Nicolás Catalano}
    \codirector{Ing. José Quinteros del Castillo}
    \carrera{Proyecto Integrador de la Carrera de Ingeniería en Telecomunicaciones}
    \grado{Estudiante}
    \laboratorio{Departamento de Ingeniería en Telecomunicaciones
    Comisión Nacional de Energía Atómica
    Centro Atómico Bariloche}
    \jurado{Alejandro Boeri (INVAP) \\ 
    Leonardo Morbidel (Instituto Balseiro)}
    \palabrasclave{formato de Tesis, Lineamientos de escritura, Instituto Balseiro}
    \keywords{Thesis format, Templates, Instituto Balseiro}
    % Si queremos poner la fecha manualmente:
    % \date{Diciembre de 2099}

\begin{document}

\begin{preliminary}

    \dedicatoria{
    A alguien
    }

    \begin{abreviaturas}
        \begin{longtable}{ll}
            ADC &   \textit{Analog-to-Digital Converter} \\
            AXI &   \textit{Advanced eXtensible Interface}  \\
            BB  &   Banda Base  \\
            BER &   \textit{Bit Error Rate} \\
            CDC &   \textit{Clock Domain Crossing}  \\
            CIAA-ACC & Computadora Abierta Industrial Argentina - Alto Costo Computacional  \\
            CLI &   \textit{Command Line Interface} \\  
            DAC &   \textit{Digital-to-Analog Converter}    \\
            DCO &   \textit{Data Clock Output}  \\
            DDC &   \textit{Digital Down Converter} \\
            DDR &   \textit{Double Data Rate}   \\
            DDS &   \textit{Direct Digital Synthesizer} \\
            DEP &   Densidad Espectral de Potencia \\   
            DOA &   \textit{Direction of Arrival}   \\
            DUT &   \textit{Device Under Test}  \\
            FF  &   \textit{Flip-Flop}  \\
            FIFO &  \textit{First In, First Out}    \\
            FIR &   \textit{Finite Impulse Response}    \\
            FFT &   \textit{Fast Fourier Transform} \\
            FPGA &  \textit{Field-Programmable Gate Array} \\
            FSR &   \textit{Full Scale Range}   \\
            FTP &   \textit{File Transfer Protocol} \\
            GIL &   \textit{Global Interpreter Lock}    \\
            GMSK &  \textit{Gaussian Minimum-Shift Keying} \\
            GUI &   \textit{Graphical User Interface}   \\
            IP &    \textit{Intellectual Property} o \textit{Internet Protocol} \\
            ITU &   \textit{International Telecommunication Union} \\
            LEO &   \textit{Low Earth Orbit}        \\
            LNA &   \textit{Low-Noise Amplifier}    \\
            LSB &   \textit{Least Significant bit}  \\
            MSK &   \textit{Minimum-Shift Keying}  \\
            NCO &   \textit{Numerically Controlled Oscillator}  \\
            OOT &   \textit{Out of Tree}    \\
            OS  &   \textit{Operating System}   \\
            PS  &   \textit{Processing System}  \\
            PL  &   \textit{Programmable Logic} \\
            RABG &  Ruido Aditivo, Blanco y Gaussiano   \\
            RF  &   Radiofrecuencia                 \\
            SDR &   \textit{Software Defined Radio} \\
            SNR &   \textit{Signal-to-Noise Ratio}  \\
            SoC &   \textit{System on a Chip}       \\
            Sps &   \textit{Samples} por segundo   \\
            SSH &   \textit{Secure Shell}   \\
            TB  &   \textit{Testbench}  \\
            TCL &   \textit{Tool Command Language}  \\
            TCP &   \textit{Transmission Control Protocol}  \\
            TLE &   \textit{Two-Line Element set}   \\
            UDP &   \textit{User Datagram Protocol} \\
            UVM &   \textit{Universal Verification Methodology} \\
            XDC &   \textit{Xilinx Design Constraints}
        \end{longtable}
    \end{abreviaturas}
    
    \tableofcontents                %Índice
    \listoffigures                  %Figuras
    \listoftables                   %Tablas
    \begin{resumen}
    Se realizó un proyecto
\end{resumen}

\begin{abstract}
    A project was carried out
\end{abstract}
    
\end{preliminary}


\subfile{Caps/Intro/intro.tex}
\subfile{Caps/Planteamiento_del_receptor/planteamiento_receptor.tex}
\subfile{Caps/Enlace/enlace.tex}
\subfile{Caps/Etapa_de_Preprocesamiento/preprocesamiento.tex}
\subfile{Caps/Integracion_con_Beamformer/beamformer.tex}
\subfile{Caps/Captura_on_Demand/streaming.tex}
\subfile{Caps/Interfaz_de_Usuario/ui.tex}
\subfile{Caps/Conclusiones/conclusiones.tex}

\appendix
\subfile{Apendices/TLE/tle.tex}
\subfile{Apendices/Doppler/doppler.tex}
\subfile{Apendices/Tipado-en-GNU/tipado-gnu.tex}
\begin{biblio}
    \bibliography{mibib}
\end{biblio}

\begin{postliminary}

    \begin{seccion}{Agradecimientos}
    A todos los que se lo merecen, por merecerlo
    \end{seccion}
    
\end{postliminary}
    
\end{document}
