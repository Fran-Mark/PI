\documentclass[../../main.tex]{subfiles}
\begin{document}

\chapter{Determinación de la cadena de acondicionamiento de RF}

La definición de la especificación de la cadena de RF es una parte fundamental en el diseño del receptor. Para todos los cálculos realizados en este capítulo se tomó como referencia la misión BEESAT-9 \cite{BEESAT-9} de la Universidad de Berlín. La razón de esta elección radica en que se trata de un satélite LEO a 530~km de altura el cual opera en la banda de interés, más específicamente a una frecuencia central $f_0 = 435.95\un{MHz}$.

\section{Planteo del front-end de RF}
La cadena de recepción propuesta se muestra en la figura FIGURA. Allí se observa que se compone de los siguientes elementos:
\begin{itemize}
    \item Un tramo de cable de aproximadamente 50~cm desde los elementos radiantes hasta el híbrido.
    \item Un híbrido en cuadratura el cual combina              
\end{itemize}

un híbrido, seguido de una etapa de filtrado, dos etapas de amplificación y un segundo filtrado.
\section{Análisis de potencia} 

\subsection{Peor caso}
\subsection{Mejor caso}

\section{Análisis de SNR}
\subsection{Requerimiento a la entrada del demodulador}
\subsection{Ganancias por procesamiento}
\subsubsection{Ganancia por filtrado}
\subsubsection{Ganancia por beamforming}
Hay ganancia mucha o no duko
\section{Determinación de la ganancia de la cadena de acondicionamiento}




\end{document}