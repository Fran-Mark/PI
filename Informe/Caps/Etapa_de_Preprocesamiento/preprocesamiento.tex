\documentclass[../../main.tex]{subfiles}
\begin{document}
\graphicspath{{./figures}}
\chapter{Preprocesamiento de los datos adquiridos}
Las señales entrantes son muestreadas por el ADC a 65~MSps en un ancho de banda de 25~MHz como se desarrolló en el \cref{sec::planteo-front-end}. Sin embargo, los \textit{beams} de salida poseen un ancho de banda de apenas 25~kHz, de manera que puede reducirse considerablemente tanto el ancho de banda como la tasa de muestreo.
\unsure{debería aclarar hasta donde -> teorema del muestreo}

Como se dijo en el \cref{sec::planteo-sist-adq}, esta sección del \textit{core} de adquisición forma parte del presente proyecto. Su diseño, implementación y validación se realizaron primero de manera aislada e independiente a lo desarrollado en \cite{proyecto-jose} y, posteriormente, se integró dentro de dicho sistema.

\section{Requerimientos de diseño}

\section{Diseño conceptual}

\subsection{Implementación en software}

\section{Implementación en hardware}
%Uso clk de 260 MHz
\subsection{IPs empleadas}
\subsection{Bloques desarrollados}
\subsection{Filtros y decimación}
\subsection{Herramientas de \textit{debug}}

\section{Verificación con UVM}
\subsection{Estructura de la simulación}
\subsection{Resultados}

\section{Integración con el sistema de adquisición}
\subsection{Adaptaciones necesarias}
%Clock a 260, CDCs
%División en varios bloques para minimizar replicas (16 canales)
%Aumento dato fifo (32 bits en lugar de 16)
%Uso de dos shorts como complex
\subsection{Otras modificaciones}
%constraints - false paths
%Uso de complemento a 2
%Raw data path para calibracion
%Cambio de índices
\subsection{Herramientas de \textit{debug}}
\end{document}